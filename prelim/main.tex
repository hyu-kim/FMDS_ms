\documentclass[12pt]{article}
\usepackage[letterpaper, margin=1in]{geometry}
\usepackage{amsfonts,amssymb,amsmath}
\usepackage{mathrsfs,mathtools}
\usepackage{comment}
\usepackage{graphicx}%
\usepackage{verbatim}
\usepackage{caption}
\usepackage{enumitem}
\usepackage{hyperref}
\usepackage{xcolor}

\newcommand{\field}[1]{\mathbb{#1}}
\newcommand{\X}{\field{X}}
\newcommand{\eps}{\varepsilon}
\newcommand{\R}{{\rm I}\kern-0.18em{\rm R}}
\newcommand{\cL}{\mathcal{L}}
\newcommand{\norm}[1]{\Vert#1\Vert}
\newcommand{\argmin}{\mathop{\mathrm{argmin}}}

\title{Dimension Reduction for Permutation-based Hypothesis Testing via factor rotation
% \textcolor{red}{Tree-distance Metric}
% \footnote{I narrowed down our scope, because I think saying ``non-Euclidean distance'' may sound too broad}
}
\author{
    Soobin Kim$^\dag$, Hyungseok Kim\thanks{Equal contribution.} , Cullen R. Buie, Hee-Seok Oh\thanks{Corresponding authors.}
    }
\date{December 2021}

\begin{document}

\maketitle

\section{Introduction}
\begin{itemize}
    \item Dimension reduction is becoming a widely used method for analyzing high-dimensional data, e.g. ecological, biological sequences, ...
    \item However, reduction towards a low dimension, to make an original data interpret-able by human, often fails to convey the original structure \textcolor{red}{\{ref1, ref2..\}}
    \item This is especially the case when the number of dimension of given dataset is very high compared to the number of samples (aka ``curse of dimensionality''). One example is  sequence analysis of biological samples, where the number of samples (replicates) less than the number of features (dimension) in a practical sense.
    \item Indeed, such failure to accurately project the original dispersion info can result in an inconsistency with quantitative analysis, such as hypothesis testing (see appendix for detail).
    \item In this study we introduce a new reduction method to overcome the inconsistency between projected data structure and hypothesis testing result. 
    % \textcolor{red}{specifically we focus on tree distance matrix, one of non-Euclidean space, and we study what their properties are in determining principle coordinates}
\end{itemize}

\section{Algorithms and main result}
Will describe soon. See appendix for details
% \subsection{Notations and preliminaries}
% \subsection{Main result}

\section{Questions}
\begin{itemize}
    \item Can we prove the existence of our solution? If so, which assumptions do we need to prove its existence?
    \item What's the time complexity of our algorithm? How fast does it converge? Can it be improved to perform better?
    \item We yet do not mention why we particularly focus on tree-based distance metric (or non Euclidean). For example, does tree-based metric help us answer any of the above? \textcolor{blue}{Two possible uses for example: 1) Use sparsity of distance matrix (I remember you mentioning that our distance matrix is sparse). 2) We may assume appropriate distribution to the data. For example, some use Markov models on trees. (Yet I'm not sure if this model is appropriate for our data)}
    \item Intuitively, I think it would be wise to start studying our algorithm with Euclidean metric, then expand its property to non-Euclidean such as phylogenic tree distances. How are Euclidean and non-Euclidean are different in their distance matrices? \textcolor{blue}{Not necessarily.. I think we should use properties of metric for proofs. Then we'll show properties of our method for general metrics, including Euclidean AND non-Euclidean. I guess generalization from Euclidean to non-Euclidean would be more difficult. (not quite certain though)}
\end{itemize}

\section{Related works}
Related previous works we've found out so far may be classified into two tracks:

\begin{enumerate}
    \item Projection Pursuit: Includes Principal Component Analysis, Factor Analysis, DiProPerm (Wei, Lee, Wichers \& Marron. 2016). In short, find the best ordination (or basis) that suits your purpose. For our data, it would be finding the best 2 or 3-dimensional projection that distinguishes two microrbial groups.
    \item Rotation based Method: Our current idea. Using existing ordination, find rotation that suits your purpose.
\end{enumerate}

I think the benefits of Rotation based Methods could be (given finding rotation matrix is feasible):
\begin{itemize}
    \item Less dependent on initial projection: Projection pursuit heavily relies on the choice of original projection, while rotation method "rotates" them (?)
    \item Interpretability: Often we give meanings to factors or PC's. We may do that to rotated ordinations. (?)
    \item Recent methods on projection pursuit heavily rely on observations (ex. \href{https://doi.org/10.1080/10618600.2015.1027773}{DiProPerm}, \href{https://arxiv.org/abs/1108.2401}{Wainwright2016}). These methods can nicely cover the addressed issue on hypothesis testing consistency. However, because the projections originate from either random process or from binary classificaion vectors. These have not yet confirmed whether the PCs can cover the covariance structure (i.e. explain variances by first few components).
\end{itemize}

% but I'm not that certain. I think we can also consider extending Projection pursuit methods as well. Also, random projection theories (especially random rotation) might come in handy.

\section{Appendix: Research context}
% this section is to provide a context to PIs how I would like to include this work in my doctoral thesis.
\section*{\small Dimension Reduction for Permutation-based Hypothesis Testing in Ecological Data}
\textit{Problem description.}\quad
High-dimensional statistics have been the most powerful tool for addressing ecological data as they present a vast size and number of samples. Interpreting these high-dimensional data is difficult, given that the data contain redundant information and are not intuitive. To extract an essential information, dimension reduction has widely been used as a tool for visualizing ecological dataset. In addition to the visualization, quantitative analysis follows to provide a statistical validation on a specific observation. Such quantitative analyses include hypothesis testing / linear regression. 

In my PhD project I have analyzed community structures of microbial samples located under two different environmental sites. After incubating the microbial cultures at each site under two different conditions (with / without their eukaryotic host), the communities were profiled by measuring their relative abundances with their 16S ribosomal RNA sequences [2].

\begin{figure}[h]
\centering
\includegraphics[width=5.7in]{images/flow.eps}
\caption{Workflow to analyze high-dimensional data in ecological studies.}
\end{figure}

Because each community sample contained 50--100 different bacterial taxa, dimension reduction (multidimensional scaling, MDS) was performed to simplify the abundance data (Fig.1B). The data were also quantitatively validated for hypothesis testing, by using permutational multivariate analysis of variance (PERMANOVA) [1], to test for a difference between two conditions imposed to the samples (with or without the host, Fig.1C). A non-Euclidean distance metric (UniFrac [3]) has been chosen to represent phylogenetic dissimilarities between community samples comprised of distinct taxonomic agents (Fig.1A).

Interestingly, MDS visualization resulted in a two-dimensional visualization with less clear difference between two conditions but with a higher \textit{F}-statistic (i.e. lower \textit{P}-value). As shown in Fig.2A, community samples in \textsf{Site 1} exhibit a more visible difference between conditions (presence of host) than those in \textsf{Site 2}. However, \textit{F}-statistic for the group difference in communities located in \textsf{Site 1} is lower than that located in \textsf{Site 2} (Fig.2B). Notably, at \textsf{Site 2} the ordination in two-dimension does not show a notable difference between the groups although it is statistically significant ($P = 0.001$). Based on the contradictory result, we hypothesized that the dimension reduction method (MDS) does not produce a visualizing result to consistently align with hypothesis testing for statistical difference between groups.
\begin{figure}[h]
\centering
\includegraphics[width=5.6in]{images/pcoa-permanova.eps}
\caption{(A) Multidimensional scaling and (B) statistical test results on a group difference between two conditions (groups).}
\end{figure}

\noindent\textit{Estimation method \& Regularization geometry.} \quad
We aim to establish a new method that can reduce the dimension while carrying over the original structure of labeled data. In specific, we adopt a factor rotation (commonly used in factor analysis) to find improved principal components, where an orthogonal rotation vector is multiplied to original principal components. To formulate our problem we first consider a binary regression model,
\begin{equation}
    Y = \sigma(\X_{B}\beta) + \eps,
\end{equation}
where $Y\in\{0,1\}^{N}$ is a binary response vector, $\sigma(\cdot)$ is a sigmoid function, $\X_B\in\R^{N\times k}$ is a design matrix which has been projected from an original design matrix $\X\in\R^{N\times d}$ onto $k$ dimension using the first $k$ vectors from an orthonormal matrix $B\in[-1,1]^{d\times d}$, and $\eps$ is a Gaussian noise vector. 

Next we consider an orthogonal rotation matrix $R\in\R^{d\times d}$, allowing us to write a revised principal component $B' = RB$ and to consider a different design matrix $\X_{B'}$ for the binary classification. We are interested in evaluating how $B'$ can well perform to classify the labeled data compared to the previous ordination $B$. For such evaluation we propose two measures that are commonly used: 1) a loss function $\cL(\X_{B'}, Y, \beta)$ which is derived from projected design matrix $\X_B'$, and 2) the \textit{F}-ratio $F(\X, Y)$ from the original design matrix $\X$. Specifically, the loss function $\cL$, writes
\begin{equation*}
    \cL(\X_{B'}, Y, \beta) = \frac{1}{2N}\norm{Y - \sigma(\X_{B'}\beta)}_2 + \lambda\norm{\beta}_1
\end{equation*}
and the latter, $F(\X, Y)$, is defined as described in Fig.1C. Furthermore, as discussed in the class we can find a Lasso estimator $\beta^*$ that minimizes the loss function, i.e.
\begin{equation*}
    \cL^* (\X_{B'}, Y, \beta^*) = \min_{\beta}\frac{1}{2N}\norm{Y - \sigma(\X_{B'}\beta)}_2 + \lambda\norm{\beta}_1,
\end{equation*}
so that we can write a pair $(\cL^*, F)$ for every given rotational matrix $R$.

Finally, we use the permutation [2] to formulate our problem into optimization. In other words, we randomly shuffle elements in response vector $Y$ to obtain a new response vector $Y^\pi$, and write a `permuted' pair $(\cL^{*\pi}, F^\pi)$ which satisfies the following:
\begin{equation*}
\begin{aligned}
    \cL^{*\pi} (\X_{B'}, Y^\pi, \beta^*) &= \min_{\beta}\frac{1}{2N}\norm{Y^\pi - \sigma(\X_{B'}\beta)}_2 + \lambda\norm{\beta}_1, \\
    F^\pi (\X, Y^\pi) &= (\text{A new \textit{F}-ratio obtained from } \X\text{ and } Y^\pi)
\end{aligned}
\end{equation*}

For a number of shuffling (e.g. $n$ = 1,000 permutations) we will obtain a sequence of pairs \newline $(\cL^{*\pi_1}, F^{\pi_1}), \cdots (\cL^{*\pi_n}, F^{\pi_n})$ which allows us to establish a linear regression model and its optimized loss function $\Tilde{\cL}^\dagger$ as
\begin{equation*}
\begin{aligned}
    F^{\pi_i} &= \gamma \cL^{*\pi_i} + \eps'_i \\
    \Tilde{\cL}^\dagger (\cL^{*\pi}, F^{\pi}) &= \min_{\gamma\in\R}\frac{1}{2n}\norm{F^\pi - \gamma\cL^{*\pi}}_2 + \lambda\norm{\beta}_1
\end{aligned}
\end{equation*}
for a given rotation matrix $R$. Now it reduces to a problem to find an optimal matrix $R^*$ such that
\begin{equation*}
    R^{*} = \argmin_{R\in\R^{d\times d}} \Tilde{\cL}^\dagger (\cL^{*\pi}, F^{\pi}).
\end{equation*}
Overall, we observe how dimension reduction (which is one representing example for parsimonious principle) can sometimes lead to an inconsistent visualization results. We have formulated the problem into a binary and linear regression settings using a Lasso estimator, to obtain an estimator matrix that can resolve the inconsistency. This formulation can be applied to improve existing dimension reduction methods, including for Euclidean and non-Euclidean distance metric settings, which will bring a broad impact in ecological studies. \newline

\small
\begin{enumerate}[leftmargin=*, nolistsep]
  \item H. Kim \textit{et al.}, Bacterial Response to Spatial Gradients of Algal-Derived Nutrients in a Porous Microplate, \textit{bioRxiv}, \href{https://www.biorxiv.org/content/10.1101/2021.06.23.449330v1.full}{doi:10.1101/2021.06.23.449330}.
  \item M. J. Anderson, A New method for Non-Parametric Multivariate Analysis of Variance of Variance, \textit{Austral Ecol.} \textbf{26}, 32--46 (2001).
  \item C. Lozupone and R. Knight, UniFrac: a New Phylogenetic Method for Comparing Microbial Communities, \textit{Appl. Environ. Microbiol.} \textbf{71}, 8228--8235 (2005).
\end{enumerate}

\end{document}
